
\chapter{Brandes' Algorithm for computing Betweenness Centrality}
\label{chap:lit}

Given a connected graph $G \equiv (V,E)$ where $V$ is total number of vertices and $E$ is total number of edges, let  $\sigma_{st}$  be the number of shortest paths from a source $s$ $\in$ $V$ to a target $t\in V$. Let $\sigma_{st} (v)$ be the number of such $s$ $\rightarrow$ $t$ paths passing through a vertex $v \in V $, $v \neq s,v \neq t$. Let the pair dependency of $v$ to <$s$, $t$> pair be
the fraction $\Delta_{st}(v) =\sigma_{st}(v) / \sigma_{st}$. The betweenness centrality of $v$ is defined by
\begin{equation} \label{eq1}
bc[v]={\sum_{s \neq v,v \neq t,s \neq t}}  \Delta_{st}(v)   
\end{equation}

Since there are $O$($n^2$) pairs in $V$, one needs $O$($n^3$) operations to compute $bc[v]$ for all $v \in V$ by using Equation (1.1). Brandes reduced this complexity and proposed an $O(mn)$ algorithm for unweighted networks. The algorithm is based on the accumulation of pair dependencies over target vertices.
After accumulation, the dependency of $v$ to $s \in V$ is 
\vspace{-0.1em}
\begin{equation} \label{eq2}
\Delta_{s}(v) = {\sum_{t \in V}}\Delta_{st}(v)
\end{equation}

Let $P_{s}(u)$ be the set of parents of $u$ on the shortest paths from $s$ to all the vertices in $V$. That is,\[  P_{s}(u) = \\{v \in V : (v, u) \in E, d_{s}(u)=d_{s}(v)+ 1\\} \]
where $d_{s}(u)$ and $d_{s}(v)$ are the shortest distances from $s$ to $u$ and $v$, respectively.
$P_{s}$ defines the shortest paths graph rooted in $s$. Brandes' observed that the accumulated dependency values can be computed recursively:
\begin{equation} \label{eq3}
\Delta_{s}(v) = \sum_{u:v \in P_{s}(u)} (\Delta_{s}(u)+1)*\sigma_{sv}/\sigma_{su}
\end{equation}
To compute $\Delta_{s}(v)$ for all $v \in V$ except $s$, Brandes' algorithm uses a two-phase approach:

First, a breadth first search (BFS) is initiated from $s$ to compute $\sigma_{sv}$ and $P_{s}(v)$ for each $v$. Then, in a back propagation phase, $\Delta_{s}(v)$ is computed for all $v \in V$ in a bottom-up manner by using Equation (1.3).
In the first phase, the algorithm computes $\sigma[v]$ for $v \in V$ which is the number of shortest paths from the source vertex $s$ to $v$. In addition, the parents of $v$ on these shortest paths are stored in $P[v]$. Before the second phase, the algorithm initializes $\delta[v]$ to 0. 
In the back propagation phase, the $\delta$  values are calculated and are added to the centrality values.

Both the phases of Algorithm 1 considers all the edges at most once, taking $O(m + n)$ time. The phases are repeated for each vertex as the source. The overall time complexity is $O(mn)$.

\begin{algorithm}
\caption{Brandes' Sequential Algorithm}

%\KwResult{Write here the result }
$bc[v] \leftarrow$ 0, $v \in V$\;
\For{$s \in V$}{
%$S$ $\leftarrow$ emptystack \;
%$P[w] \leftarrow$ empty list, $w \in V$\;
%$\sigma[t] \leftarrow$ 0, $t \in V$;  $\sigma[s] \leftarrow$ 1\;
%$d[t] \leftarrow$ -1, $t \in V$;  $d[s] \leftarrow$ 0\;
%$Q$ $\leftarrow$ emptyqueue \;
Forward phase: BFS from $s$ \\
$Q.$enqueue($s$)\;
\While{$Q$ not empty}{
$v \leftarrow Q$.dequeue()\;
$S.$push($v$)\;
\For {each neighbour $w$ of $v$}{
\If{$d[w]$ < 0}{
enqueue $w$ $\rightarrow Q$\;
$d[w] \leftarrow d[v]$ + 1\;
}
\If{$d[w] = d[v]$ + 1}{
$\sigma[w] \leftarrow \sigma[w] + \sigma[v]$\;
$P[w].$append($v$)\;
}
}
}

Backward phase: Back propagation \\
$\delta[v] \leftarrow$ 0, $v \in V$\;


 \While{$S$ not empty}{
 pop $w \leftarrow S$\; 
 \For{$v \in P[w]$}{
 $\delta[v] \leftarrow \delta[v]+ (1+\delta[w])*\sigma[v]/\sigma[w]$\;
 }
  \If{$w \neq s$}{
  $bc[w] \leftarrow bc[w]+\delta[w]$\;
   }
 }
 }
\end{algorithm}
\vspace{-1.0em}
\section{Relevance of the Work Proposed in this Thesis}
Brandes' Algorithm is by far the best known, simple and efficient algorithm for computing exact values of betweenness centrality for graphs. Various algorithms are proposed with pre-processing the graphs but this algorithm uses simplistic approach to calculate BC, In this thesis, we have considered this algorithm as benchmark for comparing our results. 
The work proposed in this thesis aims to improve execution time by proposing variants of the algorithm for trees, DAGs and graphs.


